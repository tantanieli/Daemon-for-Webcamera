%% -*- coding: utf-8 -*-
\documentclass[12pt,a4paper]{scrartcl} 
%\usepackage[14pt]{extsizes}
\usepackage[utf8]{inputenc}
\usepackage[english,russian]{babel}
\usepackage{indentfirst}
\usepackage{misccorr}
\usepackage{graphicx}
\usepackage{amsmath}
\usepackage[left=20mm, top=15mm, right=15mm, bottom=15mm, nohead, footskip=10mm]{geometry}
\begin{document}
\begin{titlepage}
  \begin{center}
    \large
    МИНИСТЕРСТВО ОБРАЗОВАНИЯ И НАУКИ \\РОССИЙСКОЙ ФЕДЕРАЦИИ\\
    федеральное государственное автономное образовательное учреждение высшего образования\\
    «САНКТ-ПЕТЕРБУРГСКИЙ ГОСУДАРСТВЕННЫЙ УНИВЕРСИТЕТ 
    АЭРОКОСМИЧЕСКОГО ПРИБОРОСТРОЕНИЯ»

    Кафедра 43   
    \end{center}
	\vfill
	\noindent КУРСОВОЙ ПРОЕКТ
    \normalsize{}\\
    \normalsize{ЗАЩИЩЕН С ОЦЕНКОЙ}\\
    \normalsize{РУКОВОДИТЕЛЬ}
    
    \underline{ст.преп}
    \hspace{5cm}
    \underline{\hspace{3cm}}
    \hspace{3cm}
    \underline{М.Д.Поляк}
    \vfill
    
	\begin{center}
	\normalsize{ПОЯСНИТЕЛЬНАЯ ЗАПИСКА}\\
	\normalsize{К КУРСОВОМУ ПРОЕКТУ}\\
	\vfill
	\normalsize{СИСТЕМА СЛЕЖЕНИЯ}\\
	\vfill
    \textsc{по дисциплине: ОПЕРАЦИОННЫЕ СИСТЕМЫ}\\
	\end{center}

	\vfill
	\noindent РАБОТУ ВЫПОЛНИЛА
	\normalsize{}\\  
	\normalsize{СТУДЕНТКА ГР.}\hspace{1cm}\underline{4331}
	\hspace{2cm}
	\underline{\hspace{3cm}}
	\hspace{3cm}
	\underline{Т.А.Белозуб}
\vfill

\begin{center}
  Санкт-Петербург, 2016 г.
\end{center}
\end{titlepage}
\newpage
%\normalsize{Цель работы: реализовать демон для веб-камеры под Linux}\\
\tableofcontents % Вывод содержания
\newpage
\section{Цель работы}
	\normalsize{Цель работы: реализовать демон для веб-камеры под ОС Linux}
\section{Задание}
	Реализовать демон для веб-камеры, который делает снимок с
	камеры каждые 30 секунд и сохраняет его на диске в одном из популярных сжатых
	графических форматов (JPEG, TIFF, и т.п.). Предусмотреть возможность настройки
	временного интервала между снимками и пути, по которому сохраняются файлы.
\section{Сравнение с аналогами}
	Есть программа, написанная на си, с похожим функционалом для ОС Linux под названием Motion. Она позволяет каждые N секунд делать снимки, позволяет пользователю указывать путь к сохраняемому файлу. Также имеется возможность выбора формата имени файла, что отсутствует в разработанном демоне. Однако данная программа лишена возможности автозапуска при загрузке системы.
\section{Техническая документация}
\subsection{Установка}
	Склонировать репозиторий с github при помощи команды: \begin{verbatim}
	git clone https://github.com/tantanieli/Daemon-for-Webcamera.git
	\end{verbatim}Для работы демона необходим установленный python версии 3.5.2 и выше.
\subsection{Использование}
	Разработанный демон поддерживает следующие команды: start, stop, status. Вызов команды осуществляется следующим образом: python main.py <команда>. Команда start запускает демона и создает PID-файл с ID процесса. Команда status сообщает о статусе демона(работает он или нет), а так же ID процесса, если демон запущен. Команда stop останавливает работу демона. 
\section{Выводы}
	В процессе выполнения данной курсовой работы мною были получены знания и навыки, необходимые для работы с вебкамерой в ОС семейства Linux, а так же знания и навыки в написании демонов.
\normalsize{Цель работы: реализовать демон для веб-камеры под Linux}
\end{document}
